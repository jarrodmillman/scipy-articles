\newpage
\section*{SciPy '02 - Python for Scientific Computing Workshop}

\begin{verbatim}
----------------------------------------
Python for Scientific Computing Workshop
----------------------------------------
CalTech, Pasadena, CA
Septemer 5-6, 2002

http://www.scipy.org/site_content/scipy02

This workshop provides a unique opportunity to learn and affect what is
happening in the realm of scientific computing with Python. Attendees will
have the opportunity to review the available tools and how they apply to
specific problems. By providing a forum for developers to share their Python
expertise with the wider industrial, academic, and research
communities, this workshop will foster collaboration and facilitate the
sharing of software components, techniques and a vision for high level
language use in scientific computing.

The two-day workshop will be a mix of invited talks and training sessions in
the morning. The afternoons will be breakout sessions with the intent of
getting standardization of tools and interfaces.

The cost of the workshop is $50.00 and includes 2 breakfasts and 2 lunches
on Sept. 5th and 6th, one dinner on Sept. 5th, and snacks during breaks.

There is a limit of 50 attendees.  Should we exceed the limit of 50
registrants, the 50 persons
selected to attend will be invited individually by the organizers.

Discussion about the conference may be directed to the SciPy-user mailing
list:

mailto:scipy-user at scipy.org

http://www.scipy.org/MailList


-------------
Co-Hosted By:
-------------

The National Biomedical Computation Resource (NBCR, SDSC, San Diego, CA)
^^^^^^^^^^^^^^^^^^^^^^^^^^^^^^^^^^^^^^^^^^^^
http://nbcr.sdsc.edu
The mission of the National Biomedical Computation Resource at the San Diego
Supercomputer Center is to conduct, catalyze, and enable biomedical research
by harnessing advanced computational technology.


The Center for Advanced Computing Research (CACR, CalTech, Pasadena, CA)
^^^^^^^^^^^^^^^^^^^^^^^^^^^^^^^^^^^^^^^^^^
http://nbcr.sdsc.edu
CACR is dedicated to the pursuit of excellence in the field of
high-performance computing, communication, and data engineering. Major
activities include carrying out large-scale scientific and engineering
applications on parallel supercomputers and coordinating collaborative
research projects on high-speed network technologies, distributed computing
and database methodologies, and related topics. Our goal is to help further
the state of the art in scientific computing.


Enthought, Inc. (Austin, TX)
^^^^^^^^^^^^^^^
http://enthought.com
Enthought, Inc. provides business and scientific computing solutions through
software development, consulting and training.

Travis N. Vaught travis at enthought.com
Mon Jun 17 19:38:52 EDT 2002
\end{verbatim}
