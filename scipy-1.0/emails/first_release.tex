\newpage
\section*{ANN: SciPy 0.10 -- Scientific Computing with Python}

\begin{verbatim}
Enthought is pleased to announce SciPy 0.10 available at:

                    www.scipy.org.

SciPy is an open source package that builds on the
strengths of Python and Numeric providing a wide range of
fast scientific and numeric functionality. SciPy's current module
set includes the following:

    Special Functions (Bessel, hanker, Airy, etc.)
    Signal/Image Processing
    2D Plotting capabilities
    Integration
    ODE solvers
    Optimization (simplex, BFGS, Netwon-CG, etc.)
    Genetic Algorithms
    Numeric -> C++ expression compiler
    Parallel programming tools
    Splines and Interpolation
    And other stuff.

Compatibility:

    SciPy relies on Python 2.1 and Numeric 20.1 (which is included
    in the binaries). It might work with other versions but nothing
    else has been tested.

License Style:

    The license is currently BSD for the 0.10 release. It could change
    to something like the Python license in the future.  Whatever the
    choice, it'll be of the "free for both non-commercial and
    commercial use, just don't sue us" style.

Mailing Lists:

        scipy-dev at scipy.org
        scipy-user at scipy.org
        searchable archives are also available at www.scipy.org

The Site:

    www.scipy.org is a community site based on Zope.  Please use its
    interactivity to host your own scientific modules, comment on pages,
    and engage in discussions (wiki-style).

Many thanks to:

    Travis Oliphant who has made huge code and infrastructure
    contributions.

    Jim Huginin, Paul Dubois, and the rest of the Numeric
    gang for building and maintaining such a powerful tool
    for scientific programming.

    The Numeric wizards that have built www.netlib.org into the
    treasure trove that it is.

    GvR and the many Python contributors for such a nice language.

Warning:

    If the 0.10 release number isn't plain enough, let me spell it
    out.  This is an alpha  release.  While much of it is very
    useable, many bugs remain.  Documentation
    exists but is still spotty.  Installation is tested on Windows
    and Linux, but still breaks occasionally.  Plotting still has
    kinks.  And many other issues.  The point of this release is
    to provide a technology preview and to solicit help with
    both finding bugs and code development.

    One other thing:  INTERFACES WILL LIKELY CHANGE!
    Function names, module names, calling conventions, etc. are
    still in flux. 0.20 will definitely not be backwards compatible
    with 0.10.  However, we'll work hard to get this stabilized as
    quickly as feasible (but not any quicker).

eric jones eric at enthought.com
Mon Aug 20 05:03:35 CEST 2001
\end{verbatim}
